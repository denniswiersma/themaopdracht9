\pagenumbering{gobble}
\newpage
\pagenumbering{Roman}

\addcontentsline{toc}{section}{Abstract}
\begin{abstract}

Heart disease is one of the most prominent problems in the world, with it demanding one of the highest death tolls in the entire world. It would therefore be of great significance to prevent this condition when early signs can be detected. To achieve such a result a dataset from the Framingham Heart Study was used to train a machine learning model to predict whether a given subject has the risk of developing coronary heart disease within the next ten years. The trained model can make such a prediction with an accuracy of over 89\%, a false negative rate of just 5\%, an f-score over 0.89, and an area under the ROC curve of over 0.94. These resuls suggest this trained model can predict the risk of developing coronary heart disease fairly accurately.

\end{abstract}
\newpage

\section*{Abbreviations}

\begin{tabular}{ll}
\textbf{ML}       & Machine Learning \\
\textbf{AI}       & Artificial Intelligence \\
\textbf{FHS}      & Framingham Heart Study \\
\textbf{CHD}      & Coronary Heart Disease \\
\textbf{SMOTE}    & Synthetic Minority Oversampling Technique \\
\textbf{FNR}      & False Negative Rate \\
\textbf{AUC-ROC}  & Area Under the Curve of Receiver Operating Characteristic \\
\textbf{SMO}      & Sequential Minimal Optimization \\
\textbf{CSC}      & Cost Sensitive Classification \\
\textbf{PCA}      & Principle Component Analysis \\
\end{tabular}

\addcontentsline{toc}{section}{Abbreviations}

\addcontentsline{toc}{section}{List of Figures}
\listoffigures 
\addcontentsline{toc}{section}{List of Tables}
\listoftables

\newpage

\tableofcontents

\newpage
\pagenumbering{arabic}